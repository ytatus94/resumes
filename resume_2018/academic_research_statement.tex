\documentclass[12pt]{article}
\usepackage{amsmath}
\usepackage{amssymb}
\usepackage{amsthm}
\usepackage{amscd}
\usepackage{amsfonts}
\usepackage{graphicx}%
\usepackage{fancyhdr}
\usepackage{authblk}
\usepackage[bottom=0.8in, top=0.5in, left=0.8in, right=0.8in]{geometry}

%\theoremstyle{plain} \numberwithin{equation}{section}
%\newtheorem{theorem}{Theorem}[section]
%\newtheorem{corollary}[theorem]{Corollary}
%\newtheorem{conjecture}{Conjecture}
%\newtheorem{lemma}[theorem]{Lemma}
%\newtheorem{proposition}[theorem]{Proposition}
%\theoremstyle{definition}
%\newtheorem{definition}[theorem]{Definition}
%\newtheorem{finalremark}[theorem]{Final Remark}
%\newtheorem{remark}[theorem]{Remark}
%\newtheorem{example}[theorem]{Example}
%\newtheorem{question}{Question} \topmargin-2cm

\title{Research Statement}
\author{Yu-Ting Shen}
\affil{Department of Physics and Astronomy\\ University of Oklahoma}
\date{\today}

\begin{document}
\maketitle

My research interest is in experimental particle physics and computing.
In particular, I am interested in applying machine learning techniques to improve the analysis results.
I received my Ph.D. degree from the University of Oklahoma (OU) in May 2018 and I am currently working as a postdoctoral researcher in the OU HEP group on supersymmetry (SUSY) searches for the Non-Universal Higgs Mass Model with 2 extra parameters (NUHM2).
For the past three years, I had the opportunity to be stationed at CERN to participate in the ATLAS collaboration to work with many excellent physicists.
I have played an important role in the electron isolation measurement and contributed to the recommendation of isolation scale factors.
All of the analysis groups in the ATLAS collaboration using isolated electrons in their study have to adopt the recommendation to calibrate the difference between the Monte Carlo simulated events and data.
My dissertation search has provided me with a unique experience working on the radiative natural SUSY which provided a possible explanation of the existence of SUSY particles while maintains the lightest Higgs mass about 125 GeV and no large cancellations at the electroweak scale.
In addition, I also have some experiences on neutrino physics and $B$-physics.

%%%%%%%%%%%%%%%%%%%%

\section{Current and Previous Research}
My Ph.D. research at the University of Oklahoma (2011 to 2018) was focused on searching for supersymmetric particles at the ATLAS detector.
After passing the written and oral Ph.D. qualifier exams, I relocated to CERN (2015 to 2018) and participated in the same-sign/3L and Higgsino LSP analysis groups.
Beside searching for SUSY particles, I also measured the electron isolation scale factors and provided the results as the ATLAS $E/\gamma$ recommendations by the Combined Performance (CP) group.

\subsection{Electron isolation efficiency}
It is very important to calibrate the simulated electron events to data collected by the ATLAS detector because almost all of the analysis groups need to consider the electrons in their studies.
The $E/\gamma$ CP group developed a Tag-and-Probe framework to measure the electron reconstruction, identification,  isolation, and trigger efficiencies.
I co-developed and maintained the measurement of the electron isolation part of the Tag-and-Probe framework.
The electron isolation efficiencies for various combinations of isolation working points and electron ID working points were measured and the data to MC scale factor ratios were calculated using $Z \to ee$ events.
The scale factors cover  a range in $E_{\textrm{T}}$ from 7 to 200 GeV and $|\eta| < 2.47$.
In 2015 and 2016, I released 5 recommendations in total and the most important two are the recommendations for Moriond and ICHEP 2016.
While working on the measurement of the electron isolation efficiencies, I also helped the di-photon group to study the isolation efficiencies using photon isolation working points.
And a journal paper at a di-photon invariant mass around 750 GeV was published~\cite{Aaboud:2016tru}.

\subsection{Same-sign 3 leptons plus jets}
With the experience in measuring the electron isolation efficiency, I worked on the real lepton efficiency measurement for the same-sign/3 leptons plus jets analysis~\cite{Aaboud:2017dmy, Aad:2016tuk}.
I developed a new framework utilizing the EventLoop algorithm which provides all advantages of object oriented programming.
The filtration function of my new framework can reduce the size of datasets by more than a factor of ten which improved the processing speed.
The automation function of my new framework provided a convenient way to calculate the real lepton efficiency and make the final tables and figures on the fly.
It took about one week of manual operations to get the 3.2 fb$^{-1}$ results using the old framework.
However, it only takes one afternoon to get the 36 fb$^{-1}$ results using my new real lepton efficiency framework.
With this great improvement, the isolation forum convenors asked me to give a presentation in their meeting and added my framework as an auxiliary tool for the isolation package.

\subsection{Non-Universal Higgs Mass model}
I worked on searching for the Higgsino LSP in NUHM2 model as my dissertation~\cite{Aaboud:2017leg}.
The NUHM2 decouples the Higgs mass doublet parameters $m^{2}_{H_{u}}$ and $m^{2}_{H_{d}}$ at the GUT scale such that $m^{2}_{H_{u}} \neq m^{2}_{H_{d}} \neq m^{2}_{0}(m_{\textrm{GUT}})$ leading to a low fine-tuning $\Delta_{EW}$ values at the electroweak scale and keeping electroweak naturalness.
The low electroweak fine-tuning (EWFT) in the NUHM2 model maintains the Higgs mass $\sim$125 GeV and $Z$ boson mass $m_{Z} = 91.2$ GeV and requires no large cancellations at the electroweak scale. 
It expects the light Higgsino masses to be 100 $\sim$ 300 GeV and the electroweak gaugino masses 300 $\sim$ 1200 GeV which are reachable by the current LHC energy scale.
We produced the MC simulations for the NUHM2, re-weighted the simplified Higgsino model to the NUHM2, and applied all the event selections, reconstruction, background suppression processes to optimize the signal.
Finally, the cross-section at next-to-leading-logarithm (NLL) accuracy was estimated using HistFitter to fit the signal and background simultaneously and an upper limit at the 95\% CL was obtained.

\subsection{Earlier research experience}
I was an active member (2009 to 2011) of TEXONO collaboration at Department of Physics in Academia Sinica.
The TEXONO experiment aimed to measure the neutrino coming from the nuclear reactor at Kuosheng Nuclear Power Plant in Taiwan.
My main duty in TEXONO was using GEANT4 package to develop the simulation program of the high purity germanium detector and studied the ambient background radiation.
I also had one month business trip to China to help Tsinghua university to build the the China Jinping Underground Laboratory (CJPL).

My Masters period (2003 to 2006) was working on the $B$-physics at the Belle experiment.
The topic of my Masters thesis is "Measurements of branching fractions and CP asymmetries in $B \to \phi \phi K$ decays at Belle"~\cite{Abe:2006qy, Shen:2008pr}.

%%%%%%%%%%%%%%%%%%%%

\section{Other Research Interests}
Because of the significant enhancement of computing power especially in the CPU and GPU processing speed, machine learning gets more attention in recent years. 
The high energy physics community starts to adopt the machine learning techniques in analyzing the experiment data sets.
There are more and more analysis groups in the ATLAS experiment using boosted decision tree (BDT) and neural network (NN) algorithms and the same situation also is happening in the CMS experiment.
For example, a NN is used to improve the background rejection and extract the signal in the single top quark analysis group lead by OU. The resulting significance is 4.2 $\sigma$ in the data~\cite{Aaboud:2017ylb}.

I am very interested in learning and applying the machine learning techniques on the  data analysis.
After graduated from OU, I spent a lot of effort in learning the knowledge and skills of  machine learning.
The data manipulation using the Python Pandas package and the machine learning using Python Scikit-learn package became part of my skills.
I have been applying the machine learning models such as linear regression, logistic regression, SVM, kNN, decision tree, and random forest to some projects and I hope these skills can be applied on the particle physics experimental data to help the analysis.

%%%%%%%%%%%%%%%%%%%%

\section{Future Directions}
%% SMU
% In the future I would like to continue working in the topic of searching for the electroweak produced SUSY particles in the compress scenario for the NUHM2 interpretation.
% This study is aimed for the summer conference and we got approval for submitting the requests to produce the official MC production a couple weeks ago.
% I would also plan to employ the machine leaning technologies to the experimental datasets to improve the analysis accuracy and precision.
%
% Furthermore I am also interested to work in the new physics beyond the Standard Model and the precision measurement of Higgs properties.
% I had some experiences of taking inner detector ACR shift and would like to  involved in the hardware upgrades such as LAr and TDAQ system.

%% Stockholm University 
% In the future I would like to continue working in the topic of searching for the electroweak produced SUSY particles in the compress scenario for the NUHM2 interpretation.
% This study is aimed for the summer conference and we got approval for submitting the requests to produce the official MC production a couple weeks ago.
% I would also plan to employ the machine leaning technologies to the experimental datasets to improve the analysis accuracy and precision.
%
% Furthermore I am also interested to work in the new physics beyond the Standard Model, or dark matter and top physics.
% I had some experiences of taking inner detector ACR shift and would like to participate in the construction, commissioning and calibration the Tile Calorimeter and the L1 Trigger system.
% I want to contribute to the combined performance groups such as E/gamma, flavour tagging and Jet/Etmiss groups.

%% BNL
In the future I would like to continue working in the topic of searching for the electroweak produced SUSY particles in the compress scenario for the NUHM2 interpretation.
This study is aimed for the summer conference and we got approval for submitting the requests to produce the official MC production a couple weeks ago.
I would also plan to leverage the machine leaning technologies to the experimental datasets to improve the analysis accuracy and precision.
Furthermore I am also interested to work in the new physics beyond the Standard Model and the software development for the data analysis using high performance computing.
%% Princeton
% I am interested in data analysis, software development, and machine learning techniques.
% I would like to work on the investigations of charged particle tracking reconstruction on parallel processor architectures and research into highly performant data analysis systems.
% I am also interested in working on developing large scale data analysis software application and applying machine learning techniques to high energy physics.
%
My background and experience in the particle physics has built a solid ground for pursuing my academic career in physics.
I am very interested in working with your group and am confident that I would be a valuable addition.

%%%%%%%%%%%%%%%%%%%%

\begin{thebibliography}{999}
\bibitem{Aaboud:2016tru} ATLAS collaboration, "Search for resonances in diphoton events at $\sqrt{s} = 13$ with the ATLAS detector", JHEP09 (2016) 001
\bibitem{Aaboud:2017dmy} ATLAS collaboration, "Search for supersymmetry in final states with two same-sign or three leptons and jets using 36 fb$^{-1}$ of $\sqrt{s} = 13$ TeV $pp$ collision data with the ATLAS detector", JHEP09 (2017) 084.
\bibitem{Aad:2016tuk} ATLAS collaboration, "Search for supersymmetry at $\sqrt{s} = 13$ TeV in final states with jets and two same-sign leptons or three leptons with the ATLAS detector", Eur. Phys. J. C 76 (2016) 259
\bibitem{Aaboud:2017leg} ATLAS collaboration, "Search for electroweak production of supersymmetric states in scenarios with compressed mass spectra at $\sqrt{s} = 13$ TeV $pp$ collision data with the ATLAS detector", PRD 97, 052010 (2018).
\bibitem{Abe:2006qy} Belle collaboration, "Observation of $B \to \phi \phi K$ Decays", arXiv:hep-ex/0609016.
\bibitem{Shen:2008pr} Belle collaboration, "Study of $B \to \phi \phi K$ Decays", arXiv:0802.1547.
\bibitem{Aaboud:2017ylb}ATLAS collaboration, "Measurement of the production cross-section of a single top quark in association with a Z boson in proton–proton collisions at 13 TeV with the ATLAS detector", Phys.Lett. B780 (2018) 557-577
\end{thebibliography}

\end{document}

