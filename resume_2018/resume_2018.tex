%%%%%%%%%%%%%%%%%%%%%%%%%%%%%%%%%%%%%%%
%
% IMPORTANT: THIS TEMPLATE NEEDS TO BE COMPILED WITH XeLaTeX
%
% This template uses several fonts not included with Windows/Linux by
% default. If you get compilation errors saying a font is missing, find the line
% on which the font is used and either change it to a font included with your
% operating system or comment the line out to use the default font.
% 
%%%%%%%%%%%%%%%%%%%%%%%%%%%%%%%%%%%%%%

\documentclass[letterpaper]{deedy-resume-openfont}


\begin{document}

%%%%%%%%%%%%%%%%%%%%%%%%%%%%%%%%%%%%%%
%
%     LAST UPDATED DATE
%
%%%%%%%%%%%%%%%%%%%%%%%%%%%%%%%%%%%%%%
\lastupdated

%%%%%%%%%%%%%%%%%%%%%%%%%%%%%%%%%%%%%%
%
%     TITLE NAME
%
%%%%%%%%%%%%%%%%%%%%%%%%%%%%%%%%%%%%%%

\namesection{Yu-Ting}{Shen}
{\urlstyle{same}
    \faPhone \ 405.200.2633\\
    \faEnvelope \ \href{mailto:YuTing.Shen-1@ou.edu}{YuTing.Shen-1@ou.edu}\\
    \faLinkedinSquare \ \href{https://www.linkedin.com/in/yu-ting-shen-6b730b160/}{linkedin.com/in/yu-ting-shen-6b730b160}\\
    \faGithub \ \href{https://github.com/ytatus94}{github.com/ytatus94}
}

%%%%%%%%%%%%%%%%%%%%%%%%%%%%%%%%%%%%%%
%     OBJECTIVE
%%%%%%%%%%%%%%%%%%%%%%%%%%%%%%%%%%%%%%

% \section{Objective}

%%%%%%%%%%%%%%%%%%%%%%%%%%%%%%%%%%%%%%
%     SKILLS
%%%%%%%%%%%%%%%%%%%%%%%%%%%%%%%%%%%%%%

\section{Skills}
% \subsection{Languages:}
% %\location{}
\raggedright{
	C/C++ \textbullet{}
	Python (pandas/numpy/scipy/scikit-learn/matplotlib/seaborn) \textbullet{}
	BASH shell script \textbullet{}
	% Git/SVN \textbullet{}
	Git \textbullet{}
	SQL \textbullet{}
	Machine learning (which one?) \textbullet{} 
	% Mac OS/Linux \textbullet{}
	% Microsoft Office/Apple Keynote/Pages \textbullet{}
	% HTML \textbullet{}
	\LaTeX\ \textbullet{}
	% Vim \textbullet{}
	% ROOT framework \textbullet{} 
	Statistics (probability, LSE, MLE, Hypothesis test, A/B test) \textbullet{}
	Data analysis (detail) \textbullet{}
	% cross department communication \textbullet{}
	% teamwork\\
}
% \subsection{Frameworks:}
% ROOT
% \subsection{Databases:}
% SQL
% \subsection{Tools:}
% Git \textbullet{} Vim
\sectionsep

%%%%%%%%%%%%%%%%%%%%%%%%%%%%%%%%%%%%%%
%     Project
%%%%%%%%%%%%%%%%%%%%%%%%%%%%%%%%%%%%%%
\section{Projects}

% \raggedright{
% 	MuscleHub AB Test \textbullet{} National park biodiversity \textbullet{} Exploring LEGO bricks \textbullet{} Discovery of handwashing \textbullet{} The GitHub history of the Scala language \textbullet{} 3-year loans
% }

% \runsubsection{Titanic}
% \descript{}
% \location{}
% \begin{tightemize}
% \item currently Top 13\%, still improving
% \end{tightemize}
% \sectionsep

\runsubsection{project name}
\descript{}
%\location{}
\begin{tightemize}
\item What I did on the project which is most important
\end{tightemize}
\sectionsep

%%%%%%%%%%%%%%%%%%%%%%%%%%%%%%%%%%%%%%
%     EXPERIENCE
%%%%%%%%%%%%%%%%%%%%%%%%%%%%%%%%%%%%%%

\section{Experience}

\runsubsection{University of Oklahoma}
\descript{| Graduate Research and Teaching Assistant}
\location{2011 - 2018 | Norman, Oklahoma}
\begin{tightemize}
\item Published 5 top journal papers and 7 conference notes, and presented in 2 conferences.
\item Relocated to CERN since 2015. Analyzed large experimental data to search for supersymmertic particles among other duties.
\item Been teaching assistant for college and graduate level courses.
% \item Been teaching assistant and helped professors to grade homework, proctor exams, and compile student achievement statistics for many courses.
% \item Dedicated to the search of supersymmertic particles.
% Used C++ and Python to implement toy Monte Carlo, cut and count, signal extraction, noise suppression, maximum likelihood, and least Chi square fit to analyze collision data within 95\% confidence level.

\end{tightemize}
\sectionsep

\runsubsection{Organisation Europ\'{e}enne pour la Recherche Nucl\'{e}aire (CERN)}
\descript{| Graduate Research Assistant}
\location{2015 - 2018 | Geneva, Switzerland}
\begin{tightemize}
\item Analyzed proton-proton collision data collected with the ATLAS detector complex.
\item Took shift in the ATLAS control room to monitor the ATLAS detector operation.
\item Calibrated the electron isolation efficiency by analyzing 30M+ events and the results were released as a central version for the entire collaboration to use.
% Mainly contributed on electron isolation measurement, real lepton efficiency measurement, and NUHM2 model.
\item Used C++ and Python with PyROOT to perform data analysis and automate the visualization of results.
\item Mined data from Grid and reduced the data size from 400 TB to $\sim$200 GB with $\sim$30M entries in total and 400+ variables per file.
\item Analyzed large dataset using methods including Monte Carlo simulation, event selection and reconstruction, background suppression (i.e. EDA, feature processing, and feature engineering). 
Applied muti-dimensional least Chi-square fit to extract signal events, performed statistical interpretations to validate model or rule out the model parameter space.
\item Developed a new object-oriented programming framework in C++ increasing the performance and accuracy of the real lepton efficiency measurement.
The new framework processes datasets 10 times larger than the previous one, automates analysis procedures reducing a week manual operations to 5 hours, and increase the efficiency to 98\%.
% \item Co-developed and maintained the C++ based Tag-and-probe framework.

% The various analysis methods such as Monte Carlo simulation, candidates extraction, noise suppression, maximum likelihood estimation are implemented.
% Finally, the least Chi-square fit and statistical interpretations are applied on the data to extract signals.

% The various analysis methods such as Monte Carlo simulation, event selection and reconstruction (data mining \& EDA), background suppression (feature processing \& feature engineering) are mainly implemented by C/C++ and the results are visualized using Python with the PyRoot package.
% The various analysis methods such as Monte Carlo simulation, event selection and reconstruction, background suppression are mainly implemented by C/C++ and the results are visualized using Python with the PyRoot package.
\end{tightemize}
\sectionsep

\runsubsection{Academia Sinica}
\descript{| Research Assistant}
\location{2009 - 2011 | Taipei, Taiwan}
% \begin{tightemize}
% \item Performed Monte Carlo simulation of the physics processes using C++ with GEANT4 package.
% \item Analyzed the kinematic distributions of the MC simulated events.
% \item Helped the Tsinghua University to setup the China Jinping Underground Laboratory.
% \end{tightemize}
\sectionsep

\runsubsection{Taiwan Semiconductor Manufacturing Company, LTD. (TSMC)}
\descript{| SPICE R\&D engineer}
\location{2006 - 2009 | Hsinchu, Taiwan}
\begin{tightemize}
\item High-voltage SPICE model data collection automation, model parameterization, model accuracy validation, model documentation, and customer support.
% \item High-voltage SPICE (Simulation Program with Integrated Circuit Emphasis) model data collection automation, model parameterization, model accuracy validation, model documentation, and customer support.
% \item Cross departments discussion about SPICE model specification, production planning and scheduling.
\end{tightemize}
\sectionsep

%%%%%%%%%%%%%%%%%%%%%%%%%%%%%%%%%%%%%%
%     EDUCATION
%%%%%%%%%%%%%%%%%%%%%%%%%%%%%%%%%%%%%%

\section{Education} 

\runsubsection{University of Oklahoma}
% \descript{| Ph.D. in Physics | Cum. GPA: 3.47 / 4.00}
\descript{| Ph.D. in Physics}
\location{2011 - 2018 | Norman, OK}
% \begin{tightemize}
% \item \href{https://hdl.handle.net/11244/299774}{\color{link}{Dissertation link}}
% \end{tightemize}
\sectionsep

\runsubsection{National Taiwan University}
% \descript{| M.S. in Physics | Cum. GPA: 3.86 / 4.00}
\descript{| M.S. in Physics}
\location{2003 - 2006 | Taipei, Taiwan}
\begin{tightemize}
% \item \href{http://www.airitilibrary.com/Publication/alDetailedMesh1?DocID=U0001-1407200616551200}{\color{link}{Thesis link}}
\item Graduate student thesis Awards, The Physical Society of Taiwan, 2006
\end{tightemize}
\sectionsep

% \runsubsection{Chung Yuan Christian University}
% \descript{| B.S. in Physics | Cum. GPA: 3.53 / 4.00}
% \location{1998 - 2002 | Chung Li, Taiwan}
% \begin{tightemize}
% \item Certificate of Holistic Achievement Award, 1998 - 1999, 1999 - 2000, and 2000 - 2001
% \item Certificate of Academic Achievement Award, 2000 - 2001
% \end{tightemize}
% \sectionsep

%%%%%%%%%%%%%%%%%%%%%%%%%%%%%%%%%%%%%%
%     PUBLICATIONS
%%%%%%%%%%%%%%%%%%%%%%%%%%%%%%%%%%%%%%
% Papers:
% 1. Electron efficiency measurements with the ATLAS detector using 2015-2016 LHC proton-proton collision data
%    https://cds.cern.ch/record/2157687
% 2. Search for electroweak production of supersymmetric states in scenarios with compressed mass spectra at $\sqrt{s} = 13$ TeV $pp$ collision data with the ATLAS detector
%    PRD 97, 052010 (2018)
%    https://journals.aps.org/prd/abstract/10.1103/PhysRevD.97.052010
% 3. Search for supersymmetry in final states with two same-sign or three leptons and jets using 36 fb$^-1$ of $\sqrt{s} = 13$ TeV $pp$ collision data with the ATLAS detector
%    JHEP09 (2017) 084
%    https://link.springer.com/article/10.1007/JHEP09(2017)084
% 4. Search for resonances in diphoton events at $\sqrt{s} = 13$ with the ATLAS detector
%    JHEP09 (2016) 001
%    https://link.springer.com/article/10.1007/JHEP09(2016)001
% 5. Search for supersymmetry at $\sqrt{s} = 13$ in final states with jets and two same-sign leptons or three leptons with the ATLAS detector
%    Eur. Phys. J. C 76 (2016) 259
%    https://link.springer.com/article/10.1140%2Fepjc%2Fs10052-016-4095-8

% Conference note
% 1. Search for scalar diphoton resonances with 15.4/fb of data collected at $\sqrt{s} = 13$ in 2015 and 2016 with the ATLAS detector
%    https://cds.cern.ch/record/2199338
% 2. Search for strongly-produced superpartners in final states with same-sign or three leptons and jets in 2015+2016 $pp$ collisions data at $\sqrt{s} = 13$ TeV
%    https://cds.cern.ch/record/2261952
% 3. Search for supersymmetry at $\sqrt{s} = 13$ TeV with two same-sign leptons or three leptons with the ATLAS detector
%    https://cds.cern.ch/record/2196580
% 4. Search for resonances in diphoton events with the ATLAS detector at $\sqrt{s} = 13$ TeV
%    https://cds.cern.ch/record/2138074
% 5. Search for supersymmetry at $\sqrt{s} = 13$ TeV in final states with jets and two same-sign leptons or three leptons with the ATLAS detector
%    https://cds.cern.ch/record/2114088
% 6. Electron efficiency measurements with the ATLAS detector using the 2015 LHC proton-proton collision data
%    https://cds.cern.ch/record/2142831
% 7. Search for supersymmetry at $\sqrt{s} = 13$ TeV in final states with jets and two same-sign leptons or three leptons with the ATLAS detector


%%%%%%%%%%%%%%%%%%%%%%%%%%%%%%%%%%%%%%
%     LEADERSHIP
%%%%%%%%%%%%%%%%%%%%%%%%%%%%%%%%%%%%%%

% \section{Leadership}
% President of Taiwanese Student Association (TSA) at the University of Oklahoma\\
% \location{2012 - 2013}
% \begin{tightemize}
% \item Organized several TSA events in campus and participated in the joint events with the Taiwanese community in Oklahoma.
% \item Led TSA to win the first place of the Eve of Nations at the University of Oklahoma.
% \end{tightemize}
% \sectionsep

\end{document}
