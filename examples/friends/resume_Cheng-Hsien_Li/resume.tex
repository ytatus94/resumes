% LaTeX file for resume 
% This file uses the resume document class (res.cls)

\documentclass[10pt]{res}
\usepackage{enumitem}
\usepackage{helvet} % Default font is the helvetica postscript font
%\usepackage{newcent} % To change the default font to the new century schoolbook postscript font uncomment this line and comment the one above
\usepackage{hyperref}
\usepackage{ragged2e} 
\usepackage{latexsym}
\usepackage[dvipsnames]{xcolor}

\setlength{\textheight}{9.5in} % increase text height to fit on 1-page 

\hypersetup{
  colorlinks   = true, %Colours links instead of ugly boxes
  urlcolor     = blue, %Colour for external hyperlinks
  linkcolor    = blue, %Colour of internal links
  citecolor   = red %Colour of citations
}

\begin{document} 

\moveleft.5\hoffset\centerline{\hfill\Large\bf Li, Cheng-Hsien}
\moveleft.5\hoffset\centerline{\hfill
\scriptsize Cell: (612) 8174996 \textbullet\, 
\scriptsize Email: \href{mailto:cheng.hsien.li.career@gmail.com}{\scriptsize cheng.hsien.li.career@gmail.com} \textbullet\,
\scriptsize LinkedIn: \href{https://www.linkedin.com/in/chenghsienli}{\scriptsize https://www.linkedin.com/in/chenghsienli}} % Your address
\moveleft\hoffset\vbox{\hrule width\resumewidth height 1pt}\smallskip                              
\begin{resume}
\justify
\vspace{-0.3 in}
\section{Summary of Qualifications}
\textbullet\,\,Strong problem-solving skills balanced with mathematical rigor and physical intuition.\\
%\textbullet\,\,Proficient in high-performance numerical computation with C/C++ language.\\
\textbullet\,\,Proficient in numerical computation and data analysis with multiple programming languages.\\
\textbullet\,\,Excellent skills in verbal and written communication with experts and non-experts.\\
\textbullet\,\,Passionate about and capable of learning new ideas and technologies in depth.
\vspace{-0.15 in}
\section{Education}
\textbullet\,\,{\bf  Ph.D. in Physics} , University of Minnesota
[GPA: 3.75/4.00 \& \href{http://hdl.handle.net/11299/191310}{Dissertation Link}]
\hfill Sep. 2017 \\
$\triangleright$ Recipient of the \textit{2011 Outstanding Teaching Award} from the School of Physics and Astronomy.\\
\textbullet\,\,{\bf  B.S. in Physics} , National Tsing Hua University, Taiwan [GPA: 3.30/4.00] \hfill 2008\\
$\triangleright$ Recipient of two research awards from the Physical Society of and National Science Council of Taiwan.
\vspace{-0.15 in}
\section{Quantitative Experience} 
\textbullet\,\,{\bf Part-time Quantitative Risk Analyst at Numeraxial LLC} \hfill since Mar. 2017\\
$\triangleright$ Prototype and document a credit value at risk calculation for corporate bonds.\\
$\triangleright$ Identify principal components of economic indicators for further study of their relation to stock returns.\\
$\triangleright$ Research, assess, and recommend quantitative methods of relevance to the firm's problems.
\vspace{-0.15 in}
\section{Programming Experience}
\textbullet\,\,{\bf C/C++}: \\ 
$\triangleright$ Implemented Monte Carlo simulations for goodness-of-fit test and maximum likelihood estimation.\\
%$\triangleright$ Simulated and visualized stochastic neutrino emission to help theorize a statistical model.\\
$\triangleright$ Developed an adaptive solver for a system of complex ODEs to study neutrino evolution problems.\\
$\triangleright$ Applied the GNU Scientific Library to performing standard and ad-hoc numerical tasks.\\
\textbullet\,\,{\bf Matlab}:\\
$\triangleright$ Implemented linear/logistic regression, principal component analysis, support vector machine, neural network, and model validation for machine learning problems (Coursera certificate: 
\href{https://www.coursera.org/account/accomplishments/verify/SR695GRCE6M2}{SR695GRCE6M2}).\\
$\triangleright$ Produced high-quality figures to visualize data from scientific computation for thesis work.\\
\textbullet\,\,{\bf Python}:\\
$\triangleright$ Automated compilation of C/C++ code or \LaTeX\, template in UNIX environment.\\
$\triangleright$ Implemented double exponential smoothing method (time series forecasting) with the SciPy stack.\\
$\triangleright$ Built an integration-preserving spline fitting function for temporal disaggregation of GDP data.
\vspace{-0.15 in}
\section{Academia Experience} 
\textbullet\,\,{\bf Graduate Research Assistant at University of Minnesota} \hfill 2013 - 2017\\
$\triangleright$ Performed statistical analysis on a sparse data set from the SN1987A observation.\\  
$\triangleright$ Derived a 3D PDE solution for neutrinos and studied its quantum-mechanical implications.\\
$\triangleright$ Received four grants to present at domestic/international conferences and first-authored two articles.\\
\textbullet\,\,{\bf Graduate Teaching Assistant at University of Minnesota}\hfill 2010 - 2013\\
$\triangleright$ Identified students' conceptual difficulty and effectively communicated with the professor.\\
$\triangleright$ Organized weekly meeting and TA duty assignment as the lead TA in the teaching team.\\
\textbullet\,\,{\bf Undergraduate Research Assistant at National Tsing Hua University} \hfill 2006 - 2008\\
$\triangleright$ Engineered a microwave component and resolved an experimental anomaly via simulation.\\ 
$\triangleright$ Coauthored three journal and conference papers and presented at National Science Council of Taiwan.
\vspace{-0.15 in}
\section{Leadership Experience} 
\textbullet\,\,{\bf Second Lieutenant in Taiwanese Air Force (compulsory service)}\hfill 2008 - 2009\\
$\triangleright$ Coordinated and prioritized the company's operation under time pressure as the duty officer.  
\vspace{-0.15 in}

\section{Other Technology Skills}
\vspace{0.07 in}
Linux, Windows, \LaTeX\ with its presentation package, Mathematica, Microsoft Office%, GitHub

\end{resume}
\end{document}