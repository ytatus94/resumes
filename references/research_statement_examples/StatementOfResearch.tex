\documentclass[a4paper]{article}
\title{Statement of Research}
\author{Irakli Chakaberia}
\date{\today}

\setlength{\topmargin}{-10mm}
\setlength{\textwidth}{7in}
\setlength{\oddsidemargin}{-8mm}
\setlength{\textheight}{9in}
\setlength{\footskip}{1in}

\begin{document}
\fontsize{12}{15}
\selectfont
\maketitle

\section{General Introduction}

At some point during my early school years it was decided that my strong skills in mathematics could be challenged further and I was admitted to the specialized physico-mathematical school and it turned out to be a successful decision. But it was my pure and deep interest towards physics as a fundamental language of nature that led me to the faculty of physics upon graduating the school. During four years of my undergraduate journey I was in the group with the theoretically inclined physics courses.

\subsection{Masters Degree}

My interests in Solid State Physics started during my bachelor's degree studies and it was an obvious decision to do my Master's in this field at Tbilisi State University. As a result, I received my masters degree for studying the properties of the wide band gap semiconductors (predominantly ZnO, ZnS) and the ways to invert their type of conductivity. I participated in the process of growing wide band gap semiconductor crystals for our research and studying different ways of crystal growth and cutting.

\subsection{PhD Degree}

At Kansas State University, I joined the Department of High Energy physics and as an introduction to the field I spent summer of 2007 at Fermilab working on the $D0$ experiment. Right after this summer I joined CMS experiment and was stationed at Fermilab. I am involved in the efforts of $V\gamma$ group studying the diboson production. My analysis is the study of helicity distributions in $Z\gamma$ production at CMS experiment at LHC.

\section{Research Statement}

Modern high energy physics experiments are a complicated synthesis of the $theory$ behind an experiment, design and development of the $detector$ to conduct the experiment, $monitoring$ every detail of the extremely complicated detector, and $analysing$ the obtained data. I had a great opportunity to work on the pixel detector of the CMS, develop tools for online monitoring of the CMS detector and data taking and of course do the interesting analysis of the data collected.

\subsection{My $Analysis$}

Large Hadron Collider was built to unravel the yet unknown parts of physics. Testing of the ElectroWeak sector of particle physics, as we know it today, can provide a great probe into the new physics. In my analysis I study the diboson production of neutral gauge bosons and look at the helicity distributions in the data obtained from hadron-hadron collisions at LHC. This analysis has not been performed on hadron colliders and is very interesting to probe the trilinear anomalous couplings. I used the helicity formalism to $theoretically$ calculate the angular distribution of the final state particles (leptons and a photon) for the $Z\gamma$ production as a result of the quark-quark scattering. Using unbinned likelihood method for the distribution function obtained I measure the helicity amplitudes for the particles involved in the process. The helicity amplitudes are very precisely predicted in the Standard Model of particle physics and any deviation of the measured results from the theoretical predictions could indicate presence of couplings between gauge bosons which are not allowed by the electroweak theory. Lepton and photon selection criteria for the data analysis is one approved by the $V\gamma$ group. Our group is measuring the $Z\gamma$ and $W\gamma$ production cross-section and setting limits on anomalous couplings between gauge bosons. I heavily contributed to the cross-check of the correctness and consistency of the entire chain of analysis in all four channels ($Z(ee)\gamma, Z(\mu\mu)\gamma, W(e\nu)\gamma, W(\mu\nu)\gamma$). Data reconstruction and event selection was done using CMS Software (CMSSW), which is the environment for CMS reconstruction and analysis tools, which uses python language for analysis configuration files and C++ as a core language.

\subsection{WebBased $Monitoring$}

My interest in programming and programming languages far precedes their need in my physics data analysis. I’ve been programming using low (basic/pascal/C) and later high level (C++/java/etc) programming languages and developing complicated algorithms starting from my school years. Software skills enabled me to quickly integrate into the efforts of the WebBased Monitoring (WBM) team. In my projects, I used C++ with ROOT libraries to construct the necessary plots. Java platform and Java servlets were used for publishing the dynamic content on the web pages. Information for the monitoring tools were fetched from different messaging systems and the Oracle database.

I developed web based tools to monitor the performance of the LHC and CMS detector. Tools that enable experts of individual subsystems to have a full, summarized and concise information in near-online regime to quickly gather the vital information and/or respond to the challenges of the detector or other supporting hardware (e.g. vacuum, power supply) or software (e.g. high level triggering). One of my early projects was so called CMS Page One, page showing current status of the detector and data collection, which is one of the few CMS technical pages that is available for public.

FillReport and DataSummary are two of my projects that are very heavily used for Run Coordination and monitoring the vast amount of details of the experiment during each LHC fill and broader period of time. These tools provide vital online as well as archival information regarding CMS and LHC performance. Each summary page provides the possibility to drill down to every detail of every LHC Fill and CMS Run.

My experience in software development and skills acquired during my work on such a big experiment as CMS, makes me confident that I can be a valuable asset in designing and developing challenging software infrastructure for high energy physics experiments.

\subsection{Hardware Experience - CMS Pixel $Detector$}

My first experience with CMS pixel detector was mere testing of forward pixel detector modules at fermilab. Later, as a part of the PIRE program, I had the opportunity to have a hands-on experience with the CMS Pixel Detector at Paul Scherrer Institute (PSI), Switzerland. While working with PSI team, I had great opportunity to observe and study the detector firsthand. I worked on the commissioning of its barrel part. During testing and commissioning I had hands on experience with all the details of the detector, architecture of individual modules, digital and analog converters, front-end electronics and software. I optimized some of the software tools (pixelOnlineSoftware, part of xDAQ Software) for testing and commissioning purposes, by implementing easier visualization schema and necessary reset options for proper initialization of the pixel detector.  After the insertion of the Pixel detector into the core of the CMS in 2008 I was very closely working with pixel detector team to properly calibrate the detector during the regular shifts at CERN.

The biggest challenge during my work at PSI was related to the detector upgrade project. Due to the high instantaneous luminosity, expected from LHC, rises the necessity for upgrade. Part of this upgrade is the increase of the digital output size from current 4 bits to 8 bits. I was assigned the task of studying the challenges of simple extension of current analog to digital converter (ADC) design to 8 bits and developing the new possible design. Together with PSI team I was able to identify the problem of the 8 bit design at hand and probable culprit. To design and test the new ADC I used electronics design software CADANCE and made quite a few steps towards the possible 8 bit design (I had an idea of utilising the current 4 bit design with adding the very precise current divider).

Working on pixel detector was a confidence building experience to tackle the detector related challenges in the future.

\section{Summary}

My diverse background in physics and broad experience in the field of high energy physics with the CMS experiment has built a solid ground for pursuing my academic career in physics. (and here probably goes the part specific to the position, something like : I am very interested in working with your group and am confident that I can contribute to the efforts of the group and etc.)


\end{document}
